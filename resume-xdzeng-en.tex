\documentclass{resume-en}

\usepackage{fontawesome5}

\newcommand{\icon}[1]{{\footnotesize#1}\,}

\hypersetup{colorlinks, urlcolor=headings}

\begin{document}

\namesection{Xiangdong}{Zeng}{
  \icon{\faEnvelope[regular]} \href{mailto:xdzeng96@gmail.com}{xdzeng96@gmail.com}
  \quad\textbullet\quad
  \icon{\faMobile*} +86 15921560740
  \quad\textbullet\quad
  \icon{\faGithub} \href{https://github.com/stone-zeng}{stone-zeng}
  \quad\textbullet\quad
  \icon{\faGlobe} \href{https://stone-zeng.site}{stone-zeng.site}
}

\section{Education}

\subsection{Fudan University, PhD in theoretical physics}[GPA: 3.45/4 \hfill Sep 2018 -- Jan 2024]

\begin{itemize}
  \item Supervisor: Prof.\ Ling-Yan Hung. Research interests: topological orders and tensor networks
  \item Relevant courses: quantum field theory, conformal field theory, general relativity, advanced electrodynamics, astrophysics, Haskell programming, etc.
  \item Scholarship: Xianghui Scholarship of Fudan University, 2018
\end{itemize}

\subsection{Fudan University, BSc in physics}[GPA: 3.57/4 \hfill Sep 2014 -- Jun 2018]

\begin{itemize}
  \item Relevant courses: quantum mechanics, statistical physics, electrodynamics, solid state physics, quantum computation and quantum information, differential geometry, group theory, etc.
  \item Scholarship: Honor Student Scholarship in Natural Science, 2018
\end{itemize}

\section{Internship experience}

\subsection{atelierAnchor}[Intern engineer \hfill Apr 2021 -- Jan 2024]

\begin{itemize}
  \item Develop websites, assist develop and testing for typefaces, participate in the editing of self-published works
\end{itemize}

\section{Research projects}

\subsection{Strange correlators and holographic tensor networks}[\hfill Jun 2020 -- Dec 2023]

\begin{itemize}
  \item The partition function of 2d critical lattice models can be obtained by taking inner product of the ground state wave-functions of topological orders and the corresponding boundary product states (i.e.\ strange correlator). Based on this idea and the fixed-point property of topological orders, the renormalization group (RG) operators can be achieved by continuously applying the \emph{F}-moves in the tensor category
  \item Renormalization group operators can provide the construction of holographic tensor networks, where we can calculate the bulk-boundary propagator and check whether it is consistent with prediction of AdS/CFT
  \item We also study the operator pushing in holographic tensor networks and give the condition that boundary operators become generalized free fields for some specific models
  \item Publications:
    Zeng \emph{et al.} \href{https://doi.org/10.3390/e25111543}{Entropy \textbf{2023}, 25(11), 1543};\enspace
    Chen \emph{et al.} \href{https://arxiv.org/abs/2210.12127}{arXiv:2210.12127}
\end{itemize}

\subsection{Tensor network representations of Virasoro and Kac--Moody algebras}[\hfill Jan 2022 -- May 2023]

\begin{itemize}
  \item We propose a general implementation of the Virasoro and Kac--Moody algebras in generic tensor network representations of 2d critical lattice models, even when the Hamiltonian is not available
  \item This method is verified numerically in various systems such as Ising model, dimer model and Fibonacci anyon system
  \item Publications:
    Zeng \emph{et al.} \href{https://doi.org/10.1103/PhysRevB.107.245146}{Phys.\ Rev.\ B \textbf{107}, 245146 (2022)};\enspace
    Wang \emph{et al.} \href{https://doi.org/10.1103/PhysRevB.106.115116}{Phys.\ Rev.\ B \textbf{106}, 115116 (2022)}
\end{itemize}

\subsection{Machine learning and Ising model}[\hfill Jan 2018 -- Jun 2018]

\begin{itemize}
  \item We use supervised learning methods such as linear classification models and neural networks composed of multiple linear layers to classify the phases of Ising model
  \item The connection between energy-based models (e.g.\ RBM) and renormalization group is explored; convolutional layers are also introduced in an attempt to theoretically establish the correspondence with AdS/CFT
\end{itemize}

\section{Skills}

\subsection{Programming}

\begin{itemize}
  \item Programming languages: Python, C/C++, JavaScript/TypeScript, Rust, Julia, Haskell, Lua
  \item Scientific computing: NumPy, SciPy, Mathematica, MATLAB
  \item Web development: HTML/CSS, Vue.js, Vite, Nuxt, Tailwind CSS, Bootstrap
  \item Others: Git, Bash, Linux, \LaTeX
\end{itemize}

\subsection{Languages}

\begin{itemize}
  \item Mandarin Chinese (native), English (fluent), Japanese (basic)
\end{itemize}

\lastupdated

\end{document}
